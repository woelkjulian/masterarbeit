\chapter{Schlussbetrachtung}
\label{cha:schlussbetrachtung}
Im Folgenden wird die gesamte Arbeit rekapituliert und ein Ausblick gegeben.

\section{Fazit}
\label{sec:fazit}
Gegenstand der vorliegenden Arbeit war die Konzeptionierung und prototypische Umsetzung eines \ac{CUI} für das Bankkonten-Management. Unter Berücksichtigung der Gebrauchstauglichkeit und den sicherheitskritischen Aspekten des Nutzungskontextes war das Ziel zu untersuchen, welche der Anforderungen zufriedenstellend erfüllt werden können. Da es sich bei \acp{CUI} um eine natürliche Art der Interaktion handelt, lag der Fokus der Konzeption für sämtliche Schritte bei den potentiellen Benutzern.\\ 
Dabei ergab sich, dass nahezu jeder der Befragten große Skepsis gegenüber \acp{CUI} zum Ausdruck brachte, da für sie vor allem Sicherheit und Datenschutz ein zentrale Rolle einnahmen. Es wurde außerdem deutlich, dass es viele der einbezogenen Personen seltsam und unnatürlich fanden mit einer Maschine zu sprechen. Andererseits konnten die angewandten Methoden aus dem Usability Engineering zeigen, dass die Vorteile eines solchen Systems durchaus von potentiellen Nutzern erkannt wurden. Anhand dieser Befunde bestätigte sich die Entscheidung viel Zeit in die Anforderungsanalyse und benutzerorientierte Konzeption zu investieren, um das nötige Grundverständnis für die Nutzer und den Nutzungskontext zu schaffen. Die Durchführung der Interviews erzielte dabei nicht das gewünschte Ergebnis, da sie nur wenig neue Informationen ergaben. Es konnte jedoch darauf geschlossen werden, dass die Fehler bei der Ausführung der Grund für die mäßigen Ergebnisse waren und nicht die Entscheidung für diese Methode. Was das Prototyping betrifft, wurde ersichtlich, wie essenziell frühe Tests bei der Entwicklung von \acp{CUI} sind. Die erarbeitete Prototyping Toolchain trug maßgeblich zum Konzept des Skills bei, wie anhand der Ergebnisse deutlich wurde. Die Umsetzung des Prototyping Systems erfüllte die gestellten Anforderungen, mit der Einschränkung, dass aufgrund des fehlenden \ac{cft} Tools anfangs viel Zeit in die Vorbereitung der Tests floss. Mit der Umsetzung des Sicherheits-Konzeptes wurde eine Möglichkeit gezeigt wie man Benutzer von Alexa auch ohne Stimmenerkennung identifizieren kann, jedoch mit dem Kompromiss eines größeren Zeitaufwandes für den Benutzer. Was die Umsetzung des Skill-Servers betrifft wurde ersichtlich, dass die implementierte Architektur eine Alternative zum Alexa \ac{SDK} darstellt. Obwohl der Einstieg etwas schwieriger war, wurde eine Möglichkeit geschaffen Redundanzen in der Verarbeitung der Slots zu vermeiden und Alexa Requests kontextbasiert zu verarbeiten. Nach der Komplettierung und Bereitstellung des Systems, zeigten erste Tests, dass der Skill gemäß der konzipierten Prozesse arbeitet. Sowohl der Kontostand-, als auch der Transaktions-Intent konnten ausgeführt werden.\\
Somit wurden die Anforderungen teilweise erfüllt. Der Kontext konnte jederzeit gewechselt werden, da es keine festen Konversationsabläufe gibt. Ebenso war das Nachfragen von kontextrelevanten Informationen durch das System gegeben. Allerdings ist das Bewusstsein für den Kontext nur eingeschränkt vorhanden. Zwar wertete der Skill Anfragen kontextbasiert aus, jedoch werden vergangene Interaktionen nicht gespeichert. Nach einem Wechsel von einer fast abgeschlossenen Transaktion zum Kontostand konnte man nicht  zurückkehren, sondern musste die Transaktion erneut starten.\\ Über die Zufriedenstellung der umgesetzten Anforderungen kann im Zuge dieser Arbeit keine fundierte Aussage getroffen werden. Um dieser Frage nachzugehen, müssen die dafür benötigten Benutzertests mit dem entwickelten Skill abseits der Arbeit durchgeführt werden. Des Weiteren ist für zukünftige Projekte zu überlegen, ob man den Prozess der Konzeption umstellt. Da die Prototyping Tests den größeren Nutzen aufwiesen, könnte man weniger Zeit in die Anforderungsanalyse und dafür mehr in das Prototyping investieren. Vor allem die konzeptionellen Ergebnisse der Arbeit werden für weitere Projekte wiederverwendet und weiterentwickelt. Darunter auch die entstandene Prototyping Toolchain, die Skill-Server Architektur und das Sicherheits-Konzept.

\section{Ausblick}
\label{sec:ausblick}
Die dargestellten Ergebnisse ließen sich durch weitere Untersuchungen ergänzen. Die Prototyping Toolchain könnte dahingehend erweitert werden, dass sie auch für das Prototyping von textbasierten \acp{CUI} Verwendung findet. Des Weiteren ist die Entwicklung des cft-Tools denkbar, um die Anwendung zu erleichtern. Auch der Skill bietet viele Möglichkeiten für weiterführende Arbeiten, wie \zB den Ausbau der vorgestellten Architektur. Wird sie vom Bankkontext entkoppelt, wäre die Veröffentlichung einer eigenen Open-Source-Bibliothek denkbar, die das Grundgerüst und möglicherweise Atomic Intents für alle Arten von Slot Typen bereitstellt. Nicht nur für Alexa, sondern auch für andere \acp{VUI} könnten Anwendungen mit der gleichen Basis entwickelt werden.\\
Was die weitere Entwicklung der \acp{CUI} im Allgemeinen betrifft, wird abschließend auf einen Absatz aus der Literatur verwiesen:
\begin{quote}
It has been predicted in a number of studies that the global market for VPAs will increase dramatically in the next few years. One factor in addition to those discussed above is the so-called cycle of increasing returns. User acceptance and adoption interact with developments in technology to produce a cycle of increasing returns. As performance improves, more people will use conversational interfaces. With more usage, there will be more data that the systems can use to learn and improve. And the more they improve, the more people will want to use them. Given this cycle, it can be expected that conversational interfaces will see a large uptake for some time to come and that this uptake will be accompanied by enhanced functionalities and performance \cite{mctear-cui}. 
\end{quote}