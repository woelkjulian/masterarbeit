\newglossaryentry{WakeWord}{
    name={Wake Word},
    description = {Bei dem \textit{Wake Word} handelt es sich um ein Wort, dass für die Aktivierung eines Alexa Endgerätes, wie \zB den Amazon Echo verwendet wird. Erst nachdem das Wake Word ausgesprochen wurde, verbindet sich das Endgerät mit dem Alexa Dienst.},
    sort=!, nonumberlist
}

\newglossaryentry{InvocationName}{
    name={Invocation Name},
    description = {Jeder Alexa Skill hat einen \textit{Invocation Name}, über den der Skill angesprochen \bzw benutzt werden kann. Häufig ist dieser identisch mit dem eigentlichen Skill Namen},
    sort=!, nonumberlist
}

\newglossaryentry{AlexaIntent}{
    name={Intent},
    description = {Semantisch ist ein Intent, die die Essenz der Konversation, oder auch die Absicht des Nutzers. Technisch gesehen handelt es sich dabei um eine Funktion eines Alexa Skills},
    sort=!, nonumberlist
}

\newglossaryentry{AlexaUtterance}{
    name={Utterance},
    description = {Utterances sind Formulierungen. Im Kontext der \acp{CUI} handelt es sich um die Formulierungen, die Benutzer für die Interaktion mit diesem verwenden können.},
    sort=!, nonumberlist
}

\newglossaryentry{AlexaSlot}{
    name={Slot},
    description = {Slots sind Teil einer Utterance und stellen dabei einen Parameter \bzw einen Platzhalter dar.},
    sort=!, nonumberlist
}


\newglossaryentry{Faas}{
    name={Function as a Service (FaaS)},
    description = {\mytodo{}{}},
    sort=!, nonumberlist
}

\newglossaryentry{LoFi}{
    name={Low Fidelity Prototyping},
    description = {Eine Form des Prototypings. Dabei hat der Prototyp wenig Ähnlichkeit mit dem Endprodukt bezüglich technischer Reife, Funktionsumfang, Funktionstiefe, Interaktivität, Darstellungstreue und Datengehalt},
    sort=!, nonumberlist
}

\newglossaryentry{MiFi}{
    name={Mid Fidelity Prototyping},
    description = {Eine Form des Prototypings. Der Prototyp hat dabei eine gewisse Ähnlichkeit mit dem Endprodukt bezüglich technischer Reife, Funktionsumfang, Funktionstiefe, Interaktivität, Darstellungstreue und Datengehalt.},
    sort=!, nonumberlist
}

\newglossaryentry{HiFi}{
    name={High Fidelity Prototyping},
    description = {Eine Form des Prototypings. Der Prototyp hat dabei große Ähnlichkeit mit dem Endprodukt bezüglich technischer Reife, Funktionsumfang, Funktionstiefe, Interaktivität, Darstellungstreue und Datengehalt.},
    sort=!, nonumberlist
}

\newglossaryentry{Iban}{
    name={International Bank Account Number},
    description = {Die \textit{International Bank Account Number (IBAN)} ist eine international anerkannte, eindeutige Kennzeichnung von Bankkonten.},
    sort=!, nonumberlist
}

\newglossaryentry{Bic}{
    name={Bank Identifier Code},
    description = {Der \textit{Bank Identifier Code (BIC)} ist ein eindeutiger Code für die Identifizierung finanzieller- und nicht finanzieller Institutionen},
    sort=!, nonumberlist
}

\newglossaryentry{Uuid}{
    name={Universally Unique Identifier},
    description = {Der \textit{Universally Unique Identifier (UUID)} ist eine 128-bit Nummer die verwendet wird, um Informationen in einem Rechner System zu identifizieren},
    sort=!, nonumberlist
}

\newglossaryentry{pushNotifications}{
    name={Push Nachrichten},
    description = {Push Benachrichtigungen sind Meldungen, die ohne das Öffnen der jeweiligen App auf dem Smartphone erscheinen. Die App muss dennoch im Hintergrund weiterlaufen. Außerdem muss eine Verbindung zum Internet bestehen. },
    sort=!, nonumberlist
}

\newglossaryentry{crud}{
    name={CRUD Operationen},
    description = {Das Akronym \textit{CRUD} steht für „create, read, update und delete“. Dabei handelt es sich um die vier Basis Funktionen eines persistenten Speichers.},
    sort=!, nonumberlist
}

\newglossaryentry{tan}{
    name={Transaction Authentication Number},
    description = {Die \textit{Transaction Authentication Number (TAN)} wird von Online Banking Diensten als ein einmalig gültiges Passwort verwendet, um Transaktionen zu autorisieren. Da sie neben der Anmeldung beim Online Dienst eine weitere Sicherheits Schicht abbilden, ist dadurch eine Zwei-Faktor-Authentifizierung gewährleistet.},
    sort=!, nonumberlist
}



