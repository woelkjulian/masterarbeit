\section*{Abstract}
\label{sec:zusammenfassung}
Mit der Veröffentlichung von \textit{Siri} im Jahr 2011 leitet \textit{Apple} eine neue Generation von \textit{\acp{VUI}} ein. Sprach- und textgesteuerte Assistenten gewinnen seitdem an Beliebtheit.\\ 
Auch wenn die Idee nichts neues ist, hat die zunehmende Rechenleistung ermöglicht, dass diese neuen Schnittstellen mit natürlicher Sprache verwendet werden können. Das lässt auf großes Potenzial in vielen Anwendungsbereichen schließen, darunter auch der Finanzsektor. Gerade als persönlicher Assistent gewährt ein solches \ac{VUI} schnellen Zugriff auf Informationen. Antworten müssen nicht mehr gesucht, sondern können direkt erfragt werden. Da sie auf natürliche Sprache reagieren, ist es theoretisch auch nicht mehr nötig, die Sprachbefehle für die Interaktion zu kennen. Jedoch verstehen diese Technologien nur soviel, wie man ihnen beibringt. Damit sie im Stande sind, echte konversationsnahe Unterhaltungen zu führen, müssen sie dementsprechend entwickelt werden. Weg vom einfachen Frage-Antwort \textit{Voice User Interface}, hin zum \textit{\ac{CUI}}.\\
Die Arbeit umfasst zum einen die Konzeption eines solchen Systems. Es wird festgestellt, welche Anforderungen an ein \ac{CUI} im Allgemeinen und an eine entsprechende Anwendung für das Bankkonten-Management gestellt werden. Um das zu erreichen, werden Paradigmen und Methoden aus dem Bereich des \textit{Usability Engineering} angewandt. Im Zuge dessen wurde auch eine Prototyping Toolchain für \acp{CUI} entwickelt. Zum anderen soll auf Basis dessen ein prototypisches System mit \textit{Amazon Alexa} \cite{amazon-developer-alexa} umgesetzt werden.\\
Das Ziel der Arbeit ist festzustellen, welche der Anforderungen mit dem entwickelten Prototypen zufriedenstellend erfüllt werden können. Bei der Umsetzung wurde nicht das von Amazon bereitgestellte \ac{SDK} \cite{alexa-sdk} verwendet, sondern eine eigene Lösung erarbeitet. Implementiert wurden dabei die eigentliche Anwendung (Alexa Skill), ein Backend mit angebundener Datenbank und eine Smartphone-Anwendung. Der Skill, das Backend und die Datenbank wurden über externe Server bereitgestellt. Benutzer können in Verbindung mit einem Alexa Endgerät und der entwickelten App die implementierten Funktionen für fiktive Konten nutzen. Dabei wurden für den Prototypen das Abfragen des Kontostandes und die Durchführung von Transaktionen umgesetzt. 